\documentclass{article}

\usepackage{fancyhdr}
\usepackage{extramarks}
\usepackage{amsmath}
\usepackage{amsthm}
\usepackage{amsfonts}
\usepackage{tikz}
\usepackage[plain]{algorithm}
\usepackage{algpseudocode}
\usepackage[round]{natbib}
\usepackage{hyperref}
\usepackage{amssymb}




%%%%% NEW MATH DEFINITIONS %%%%%

\usepackage{amsmath,amsfonts,bm}

% Mark sections of captions for referring to divisions of figures
\newcommand{\figleft}{{\em (Left)}}
\newcommand{\figcenter}{{\em (Center)}}
\newcommand{\figright}{{\em (Right)}}
\newcommand{\figtop}{{\em (Top)}}
\newcommand{\figbottom}{{\em (Bottom)}}
\newcommand{\captiona}{{\em (a)}}
\newcommand{\captionb}{{\em (b)}}
\newcommand{\captionc}{{\em (c)}}
\newcommand{\captiond}{{\em (d)}}

% Highlight a newly defined term
\newcommand{\newterm}[1]{{\bf #1}}


% Figure reference, lower-case.
\def\figref#1{figure~\ref{#1}}
% Figure reference, capital. For start of sentence
\def\Figref#1{Figure~\ref{#1}}
\def\twofigref#1#2{figures \ref{#1} and \ref{#2}}
\def\quadfigref#1#2#3#4{figures \ref{#1}, \ref{#2}, \ref{#3} and \ref{#4}}
% Section reference, lower-case.
\def\secref#1{section~\ref{#1}}
% Section reference, capital.
\def\Secref#1{Section~\ref{#1}}
% Reference to two sections.
\def\twosecrefs#1#2{sections \ref{#1} and \ref{#2}}
% Reference to three sections.
\def\secrefs#1#2#3{sections \ref{#1}, \ref{#2} and \ref{#3}}
% Reference to an equation, lower-case.
\def\eqref#1{equation~\ref{#1}}
% Reference to an equation, upper case
\def\Eqref#1{Equation~\ref{#1}}
% A raw reference to an equation---avoid using if possible
\def\plaineqref#1{\ref{#1}}
% Reference to a chapter, lower-case.
\def\chapref#1{chapter~\ref{#1}}
% Reference to an equation, upper case.
\def\Chapref#1{Chapter~\ref{#1}}
% Reference to a range of chapters
\def\rangechapref#1#2{chapters\ref{#1}--\ref{#2}}
% Reference to an algorithm, lower-case.
\def\algref#1{algorithm~\ref{#1}}
% Reference to an algorithm, upper case.
\def\Algref#1{Algorithm~\ref{#1}}
\def\twoalgref#1#2{algorithms \ref{#1} and \ref{#2}}
\def\Twoalgref#1#2{Algorithms \ref{#1} and \ref{#2}}
% Reference to a part, lower case
\def\partref#1{part~\ref{#1}}
% Reference to a part, upper case
\def\Partref#1{Part~\ref{#1}}
\def\twopartref#1#2{parts \ref{#1} and \ref{#2}}

\def\ceil#1{\lceil #1 \rceil}
\def\floor#1{\lfloor #1 \rfloor}
\def\1{\bm{1}}
\newcommand{\train}{\mathcal{D}}
\newcommand{\valid}{\mathcal{D_{\mathrm{valid}}}}
\newcommand{\test}{\mathcal{D_{\mathrm{test}}}}

\def\eps{{\epsilon}}


% Random variables
\def\reta{{\textnormal{$\eta$}}}
\def\ra{{\textnormal{a}}}
\def\rb{{\textnormal{b}}}
\def\rc{{\textnormal{c}}}
\def\rd{{\textnormal{d}}}
\def\re{{\textnormal{e}}}
\def\rf{{\textnormal{f}}}
\def\rg{{\textnormal{g}}}
\def\rh{{\textnormal{h}}}
\def\ri{{\textnormal{i}}}
\def\rj{{\textnormal{j}}}
\def\rk{{\textnormal{k}}}
\def\rl{{\textnormal{l}}}
% rm is already a command, just don't name any random variables m
\def\rn{{\textnormal{n}}}
\def\ro{{\textnormal{o}}}
\def\rp{{\textnormal{p}}}
\def\rq{{\textnormal{q}}}
\def\rr{{\textnormal{r}}}
\def\rs{{\textnormal{s}}}
\def\rt{{\textnormal{t}}}
\def\ru{{\textnormal{u}}}
\def\rv{{\textnormal{v}}}
\def\rw{{\textnormal{w}}}
\def\rx{{\textnormal{x}}}
\def\ry{{\textnormal{y}}}
\def\rz{{\textnormal{z}}}

% Random vectors
\def\rvepsilon{{\mathbf{\epsilon}}}
\def\rvtheta{{\mathbf{\theta}}}
\def\rva{{\mathbf{a}}}
\def\rvb{{\mathbf{b}}}
\def\rvc{{\mathbf{c}}}
\def\rvd{{\mathbf{d}}}
\def\rve{{\mathbf{e}}}
\def\rvf{{\mathbf{f}}}
\def\rvg{{\mathbf{g}}}
\def\rvh{{\mathbf{h}}}
\def\rvu{{\mathbf{i}}}
\def\rvj{{\mathbf{j}}}
\def\rvk{{\mathbf{k}}}
\def\rvl{{\mathbf{l}}}
\def\rvm{{\mathbf{m}}}
\def\rvn{{\mathbf{n}}}
\def\rvo{{\mathbf{o}}}
\def\rvp{{\mathbf{p}}}
\def\rvq{{\mathbf{q}}}
\def\rvr{{\mathbf{r}}}
\def\rvs{{\mathbf{s}}}
\def\rvt{{\mathbf{t}}}
\def\rvu{{\mathbf{u}}}
\def\rvv{{\mathbf{v}}}
\def\rvw{{\mathbf{w}}}
\def\rvx{{\mathbf{x}}}
\def\rvy{{\mathbf{y}}}
\def\rvz{{\mathbf{z}}}

% Elements of random vectors
\def\erva{{\textnormal{a}}}
\def\ervb{{\textnormal{b}}}
\def\ervc{{\textnormal{c}}}
\def\ervd{{\textnormal{d}}}
\def\erve{{\textnormal{e}}}
\def\ervf{{\textnormal{f}}}
\def\ervg{{\textnormal{g}}}
\def\ervh{{\textnormal{h}}}
\def\ervi{{\textnormal{i}}}
\def\ervj{{\textnormal{j}}}
\def\ervk{{\textnormal{k}}}
\def\ervl{{\textnormal{l}}}
\def\ervm{{\textnormal{m}}}
\def\ervn{{\textnormal{n}}}
\def\ervo{{\textnormal{o}}}
\def\ervp{{\textnormal{p}}}
\def\ervq{{\textnormal{q}}}
\def\ervr{{\textnormal{r}}}
\def\ervs{{\textnormal{s}}}
\def\ervt{{\textnormal{t}}}
\def\ervu{{\textnormal{u}}}
\def\ervv{{\textnormal{v}}}
\def\ervw{{\textnormal{w}}}
\def\ervx{{\textnormal{x}}}
\def\ervy{{\textnormal{y}}}
\def\ervz{{\textnormal{z}}}

% Random matrices
\def\rmA{{\mathbf{A}}}
\def\rmB{{\mathbf{B}}}
\def\rmC{{\mathbf{C}}}
\def\rmD{{\mathbf{D}}}
\def\rmE{{\mathbf{E}}}
\def\rmF{{\mathbf{F}}}
\def\rmG{{\mathbf{G}}}
\def\rmH{{\mathbf{H}}}
\def\rmI{{\mathbf{I}}}
\def\rmJ{{\mathbf{J}}}
\def\rmK{{\mathbf{K}}}
\def\rmL{{\mathbf{L}}}
\def\rmM{{\mathbf{M}}}
\def\rmN{{\mathbf{N}}}
\def\rmO{{\mathbf{O}}}
\def\rmP{{\mathbf{P}}}
\def\rmQ{{\mathbf{Q}}}
\def\rmR{{\mathbf{R}}}
\def\rmS{{\mathbf{S}}}
\def\rmT{{\mathbf{T}}}
\def\rmU{{\mathbf{U}}}
\def\rmV{{\mathbf{V}}}
\def\rmW{{\mathbf{W}}}
\def\rmX{{\mathbf{X}}}
\def\rmY{{\mathbf{Y}}}
\def\rmZ{{\mathbf{Z}}}

% Elements of random matrices
\def\ermA{{\textnormal{A}}}
\def\ermB{{\textnormal{B}}}
\def\ermC{{\textnormal{C}}}
\def\ermD{{\textnormal{D}}}
\def\ermE{{\textnormal{E}}}
\def\ermF{{\textnormal{F}}}
\def\ermG{{\textnormal{G}}}
\def\ermH{{\textnormal{H}}}
\def\ermI{{\textnormal{I}}}
\def\ermJ{{\textnormal{J}}}
\def\ermK{{\textnormal{K}}}
\def\ermL{{\textnormal{L}}}
\def\ermM{{\textnormal{M}}}
\def\ermN{{\textnormal{N}}}
\def\ermO{{\textnormal{O}}}
\def\ermP{{\textnormal{P}}}
\def\ermQ{{\textnormal{Q}}}
\def\ermR{{\textnormal{R}}}
\def\ermS{{\textnormal{S}}}
\def\ermT{{\textnormal{T}}}
\def\ermU{{\textnormal{U}}}
\def\ermV{{\textnormal{V}}}
\def\ermW{{\textnormal{W}}}
\def\ermX{{\textnormal{X}}}
\def\ermY{{\textnormal{Y}}}
\def\ermZ{{\textnormal{Z}}}

% Vectors
\def\vzero{{\bm{0}}}
\def\vone{{\bm{1}}}
\def\vmu{{\bm{\mu}}}
\def\vtheta{{\bm{\theta}}}
\def\va{{\bm{a}}}
\def\vb{{\bm{b}}}
\def\vc{{\bm{c}}}
\def\vd{{\bm{d}}}
\def\ve{{\bm{e}}}
\def\vf{{\bm{f}}}
\def\vg{{\bm{g}}}
\def\vh{{\bm{h}}}
\def\vi{{\bm{i}}}
\def\vj{{\bm{j}}}
\def\vk{{\bm{k}}}
\def\vl{{\bm{l}}}
\def\vm{{\bm{m}}}
\def\vn{{\bm{n}}}
\def\vo{{\bm{o}}}
\def\vp{{\bm{p}}}
\def\vq{{\bm{q}}}
\def\vr{{\bm{r}}}
\def\vs{{\bm{s}}}
\def\vt{{\bm{t}}}
\def\vu{{\bm{u}}}
\def\vv{{\bm{v}}}
\def\vw{{\bm{w}}}
\def\vx{{\bm{x}}}
\def\vy{{\bm{y}}}
\def\vz{{\bm{z}}}

% Elements of vectors
\def\evalpha{{\alpha}}
\def\evbeta{{\beta}}
\def\evepsilon{{\epsilon}}
\def\evlambda{{\lambda}}
\def\evomega{{\omega}}
\def\evmu{{\mu}}
\def\evpsi{{\psi}}
\def\evsigma{{\sigma}}
\def\evtheta{{\theta}}
\def\eva{{a}}
\def\evb{{b}}
\def\evc{{c}}
\def\evd{{d}}
\def\eve{{e}}
\def\evf{{f}}
\def\evg{{g}}
\def\evh{{h}}
\def\evi{{i}}
\def\evj{{j}}
\def\evk{{k}}
\def\evl{{l}}
\def\evm{{m}}
\def\evn{{n}}
\def\evo{{o}}
\def\evp{{p}}
\def\evq{{q}}
\def\evr{{r}}
\def\evs{{s}}
\def\evt{{t}}
\def\evu{{u}}
\def\evv{{v}}
\def\evw{{w}}
\def\evx{{x}}
\def\evy{{y}}
\def\evz{{z}}

% Matrix
\def\mA{{\bm{A}}}
\def\mB{{\bm{B}}}
\def\mC{{\bm{C}}}
\def\mD{{\bm{D}}}
\def\mE{{\bm{E}}}
\def\mF{{\bm{F}}}
\def\mG{{\bm{G}}}
\def\mH{{\bm{H}}}
\def\mI{{\bm{I}}}
\def\mJ{{\bm{J}}}
\def\mK{{\bm{K}}}
\def\mL{{\bm{L}}}
\def\mM{{\bm{M}}}
\def\mN{{\bm{N}}}
\def\mO{{\bm{O}}}
\def\mP{{\bm{P}}}
\def\mQ{{\bm{Q}}}
\def\mR{{\bm{R}}}
\def\mS{{\bm{S}}}
\def\mT{{\bm{T}}}
\def\mU{{\bm{U}}}
\def\mV{{\bm{V}}}
\def\mW{{\bm{W}}}
\def\mX{{\bm{X}}}
\def\mY{{\bm{Y}}}
\def\mZ{{\bm{Z}}}
\def\mBeta{{\bm{\beta}}}
\def\mPhi{{\bm{\Phi}}}
\def\mLambda{{\bm{\Lambda}}}
\def\mSigma{{\bm{\Sigma}}}

% Tensor
\DeclareMathAlphabet{\mathsfit}{\encodingdefault}{\sfdefault}{m}{sl}
\SetMathAlphabet{\mathsfit}{bold}{\encodingdefault}{\sfdefault}{bx}{n}
\newcommand{\tens}[1]{\bm{\mathsfit{#1}}}
\def\tA{{\tens{A}}}
\def\tB{{\tens{B}}}
\def\tC{{\tens{C}}}
\def\tD{{\tens{D}}}
\def\tE{{\tens{E}}}
\def\tF{{\tens{F}}}
\def\tG{{\tens{G}}}
\def\tH{{\tens{H}}}
\def\tI{{\tens{I}}}
\def\tJ{{\tens{J}}}
\def\tK{{\tens{K}}}
\def\tL{{\tens{L}}}
\def\tM{{\tens{M}}}
\def\tN{{\tens{N}}}
\def\tO{{\tens{O}}}
\def\tP{{\tens{P}}}
\def\tQ{{\tens{Q}}}
\def\tR{{\tens{R}}}
\def\tS{{\tens{S}}}
\def\tT{{\tens{T}}}
\def\tU{{\tens{U}}}
\def\tV{{\tens{V}}}
\def\tW{{\tens{W}}}
\def\tX{{\tens{X}}}
\def\tY{{\tens{Y}}}
\def\tZ{{\tens{Z}}}


% Graph
\def\gA{{\mathcal{A}}}
\def\gB{{\mathcal{B}}}
\def\gC{{\mathcal{C}}}
\def\gD{{\mathcal{D}}}
\def\gE{{\mathcal{E}}}
\def\gF{{\mathcal{F}}}
\def\gG{{\mathcal{G}}}
\def\gH{{\mathcal{H}}}
\def\gI{{\mathcal{I}}}
\def\gJ{{\mathcal{J}}}
\def\gK{{\mathcal{K}}}
\def\gL{{\mathcal{L}}}
\def\gM{{\mathcal{M}}}
\def\gN{{\mathcal{N}}}
\def\gO{{\mathcal{O}}}
\def\gP{{\mathcal{P}}}
\def\gQ{{\mathcal{Q}}}
\def\gR{{\mathcal{R}}}
\def\gS{{\mathcal{S}}}
\def\gT{{\mathcal{T}}}
\def\gU{{\mathcal{U}}}
\def\gV{{\mathcal{V}}}
\def\gW{{\mathcal{W}}}
\def\gX{{\mathcal{X}}}
\def\gY{{\mathcal{Y}}}
\def\gZ{{\mathcal{Z}}}

% Sets
\def\sA{{\mathbb{A}}}
\def\sB{{\mathbb{B}}}
\def\sC{{\mathbb{C}}}
\def\sD{{\mathbb{D}}}
% Don't use a set called E, because this would be the same as our symbol
% for expectation.
\def\sF{{\mathbb{F}}}
\def\sG{{\mathbb{G}}}
\def\sH{{\mathbb{H}}}
\def\sI{{\mathbb{I}}}
\def\sJ{{\mathbb{J}}}
\def\sK{{\mathbb{K}}}
\def\sL{{\mathbb{L}}}
\def\sM{{\mathbb{M}}}
\def\sN{{\mathbb{N}}}
\def\sO{{\mathbb{O}}}
\def\sP{{\mathbb{P}}}
\def\sQ{{\mathbb{Q}}}
\def\sR{{\mathbb{R}}}
\def\sS{{\mathbb{S}}}
\def\sT{{\mathbb{T}}}
\def\sU{{\mathbb{U}}}
\def\sV{{\mathbb{V}}}
\def\sW{{\mathbb{W}}}
\def\sX{{\mathbb{X}}}
\def\sY{{\mathbb{Y}}}
\def\sZ{{\mathbb{Z}}}

% Entries of a matrix
\def\emLambda{{\Lambda}}
\def\emA{{A}}
\def\emB{{B}}
\def\emC{{C}}
\def\emD{{D}}
\def\emE{{E}}
\def\emF{{F}}
\def\emG{{G}}
\def\emH{{H}}
\def\emI{{I}}
\def\emJ{{J}}
\def\emK{{K}}
\def\emL{{L}}
\def\emM{{M}}
\def\emN{{N}}
\def\emO{{O}}
\def\emP{{P}}
\def\emQ{{Q}}
\def\emR{{R}}
\def\emS{{S}}
\def\emT{{T}}
\def\emU{{U}}
\def\emV{{V}}
\def\emW{{W}}
\def\emX{{X}}
\def\emY{{Y}}
\def\emZ{{Z}}
\def\emSigma{{\Sigma}}

% entries of a tensor
% Same font as tensor, without \bm wrapper
\newcommand{\etens}[1]{\mathsfit{#1}}
\def\etLambda{{\etens{\Lambda}}}
\def\etA{{\etens{A}}}
\def\etB{{\etens{B}}}
\def\etC{{\etens{C}}}
\def\etD{{\etens{D}}}
\def\etE{{\etens{E}}}
\def\etF{{\etens{F}}}
\def\etG{{\etens{G}}}
\def\etH{{\etens{H}}}
\def\etI{{\etens{I}}}
\def\etJ{{\etens{J}}}
\def\etK{{\etens{K}}}
\def\etL{{\etens{L}}}
\def\etM{{\etens{M}}}
\def\etN{{\etens{N}}}
\def\etO{{\etens{O}}}
\def\etP{{\etens{P}}}
\def\etQ{{\etens{Q}}}
\def\etR{{\etens{R}}}
\def\etS{{\etens{S}}}
\def\etT{{\etens{T}}}
\def\etU{{\etens{U}}}
\def\etV{{\etens{V}}}
\def\etW{{\etens{W}}}
\def\etX{{\etens{X}}}
\def\etY{{\etens{Y}}}
\def\etZ{{\etens{Z}}}

% The true underlying data generating distribution
\newcommand{\pdata}{p_{\rm{data}}}
% The empirical distribution defined by the training set
\newcommand{\ptrain}{\hat{p}_{\rm{data}}}
\newcommand{\Ptrain}{\hat{P}_{\rm{data}}}
% The model distribution
\newcommand{\pmodel}{p_{\rm{model}}}
\newcommand{\Pmodel}{P_{\rm{model}}}
\newcommand{\ptildemodel}{\tilde{p}_{\rm{model}}}
% Stochastic autoencoder distributions
\newcommand{\pencode}{p_{\rm{encoder}}}
\newcommand{\pdecode}{p_{\rm{decoder}}}
\newcommand{\precons}{p_{\rm{reconstruct}}}

\newcommand{\laplace}{\mathrm{Laplace}} % Laplace distribution

\newcommand{\E}{\mathbb{E}}
\newcommand{\Ls}{\mathcal{L}}
\newcommand{\R}{\mathbb{R}}
\newcommand{\emp}{\tilde{p}}
\newcommand{\lr}{\alpha}
\newcommand{\reg}{\lambda}
\newcommand{\rect}{\mathrm{rectifier}}
\newcommand{\softmax}{\mathrm{softmax}}
\newcommand{\sigmoid}{\sigma}
\newcommand{\softplus}{\zeta}
\newcommand{\KL}{D_{\mathrm{KL}}}
\newcommand{\Var}{\mathrm{Var}}
\newcommand{\standarderror}{\mathrm{SE}}
\newcommand{\Cov}{\mathrm{Cov}}
% Wolfram Mathworld says $L^2$ is for function spaces and $\ell^2$ is for vectors
% But then they seem to use $L^2$ for vectors throughout the site, and so does
% wikipedia.
\newcommand{\normlzero}{L^0}
\newcommand{\normlone}{L^1}
\newcommand{\normltwo}{L^2}
\newcommand{\normlp}{L^p}
\newcommand{\normmax}{L^\infty}

\newcommand{\parents}{Pa} % See usage in notation.tex. Chosen to match Daphne's book.

\DeclareMathOperator*{\argmax}{arg\,max}
\DeclareMathOperator*{\argmin}{arg\,min}

\DeclareMathOperator{\sign}{sign}
\DeclareMathOperator{\Tr}{Tr}
\let\ab\allowbreak

\usetikzlibrary{automata,positioning}

%
% Basic Document Settings
%

\topmargin=-0.45in
\evensidemargin=0in
\oddsidemargin=0in
\textwidth=6.5in
\textheight=9.0in
\headsep=0.25in

\linespread{1.1}

\pagestyle{fancy}
\lhead{\hmwkAuthorName}
\chead{\hmwkClass\ }
\rhead{\firstxmark}
\lfoot{\lastxmark}
\cfoot{\thepage}

\renewcommand\headrulewidth{0.4pt}
\renewcommand\footrulewidth{0.4pt}

\setlength\parindent{0pt}

%
% Create Problem Sections
%

\newcommand{\enterProblemHeader}[1]{
    \nobreak\extramarks{}{Problem \arabic{#1} continued on next page\ldots}\nobreak{}
    \nobreak\extramarks{Problem \arabic{#1} (continued)}{Problem \arabic{#1} continued on next page\ldots}\nobreak{}
}

\newcommand{\exitProblemHeader}[1]{
    \nobreak\extramarks{Problem \arabic{#1} (continued)}{Problem \arabic{#1} continued on next page\ldots}\nobreak{}
    \stepcounter{#1}
    \nobreak\extramarks{Problem \arabic{#1}}{}\nobreak{}
}

\setcounter{secnumdepth}{0}
\newcounter{partCounter}
\newcounter{homeworkProblemCounter}
\setcounter{homeworkProblemCounter}{1}
\nobreak\extramarks{Problem \arabic{homeworkProblemCounter}}{}\nobreak{}

%
% Homework Problem Environment
%
% This environment takes an optional argument. When given, it will adjust the
% problem counter. This is useful for when the problems given for your
% assignment aren't sequential. See the last 3 problems of this template for an
% example.
%
\newenvironment{homeworkProblem}[1][-1]{
    \ifnum#1>0
        \setcounter{homeworkProblemCounter}{#1}
    \fi
    \section{Problem \arabic{homeworkProblemCounter}}
    \setcounter{partCounter}{1}
    \enterProblemHeader{homeworkProblemCounter}
}{
    \exitProblemHeader{homeworkProblemCounter}
}

%
% Homework Details
%   - Title
%   - Due date
%   - Class
%   - Section/Time
%   - Instructor
%   - Author
%

\newcommand{\hmwkTitle}{Homework 1}
\newcommand{\hmwkDueDate}{February 7, 2015}
\newcommand{\hmwkClass}{CS395T Matrix and Tensor Algorithms}
\newcommand{\hmwkClassTime}{Section A}
\newcommand{\hmwkClassInstructor}{Professor Shashanka Ubaru}
\newcommand{\hmwkAuthorName}{}

%
% Title Page
%

% \title{
%     \vspace{2in}
%     \textmd{\textbf{\hmwkClass:\ \hmwkTitle}}\\
%     \normalsize\vspace{0.1in}\small{Due\ on\ \hmwkDueDate\ at 3:10pm}\\
%     \vspace{0.1in}\large{\textit{\hmwkClassInstructor\ \hmwkClassTime}}
%     \vspace{3in}
% }

% \author{\hmwkAuthorName}
% \date{}

\renewcommand{\part}[1]{\textbf{\large Part \Alph{partCounter}}\stepcounter{partCounter}\\}

%
% Various Helper Commands
%

% Useful for algorithms
\newcommand{\alg}[1]{\textsc{\bfseries \footnotesize #1}}

% For derivatives
\newcommand{\deriv}[1]{\frac{\mathrm{d}}{\mathrm{d}x} (#1)}

% For partial derivatives
\newcommand{\pderiv}[2]{\frac{\partial}{\partial #1} (#2)}

% Integral dx
\newcommand{\dx}{\mathrm{d}x}

% Alias for the Solution section header
\newcommand{\solution}{\textbf{\large Solution}}

% Probability commands: Expectation, Variance, Covariance, Bias

\newcommand{\Bias}{\mathrm{Bias}}

\begin{document}

\begin{homeworkProblem}
    \solution

    We are given an \(n \times d\) matrix \(\mathbf{A}\) with singular value decomposition
\[
\mathbf{A} = \mathbf{U}\Sigma \mathbf{V}^\top,
\]
where \(\Sigma = \operatorname{diag}(\sigma_1,\sigma_2,\dots,\sigma_d)\) with \(\sigma_1 \ge \sigma_2 \ge \cdots \ge \sigma_d\), and \(\mathbf{v}_1,\mathbf{v}_2,\dots,\mathbf{v}_d\) denote the right singular vectors. Let \(\mathbf{x}\) be a unit vector satisfying
\[
|\mathbf{x}^\top \mathbf{v}_1| \ge \eta,
\]
and write \(\mathbf{x}\) in the \(\{\mathbf{v}_i\}\) basis:
\[
\mathbf{x} = \sum_{i=1}^d \alpha_i \mathbf{v}_i, \quad \text{with } \sum_{i=1}^d \alpha_i^2 = 1 \quad \text{and } |\alpha_1| \ge \eta.
\]

After \(q\) iterations of the power method, the iterate is
\[
\mathbf{z} = \frac{(\mathbf{A}^\top \mathbf{A})^q \mathbf{x}}{\|(\mathbf{A}^\top \mathbf{A})^q \mathbf{x}\|}.
\]
Since \(\mathbf{A}^\top \mathbf{A} = \mathbf{V}\Sigma^2 \mathbf{V}^\top\), it follows that
\[
(\mathbf{A}^\top \mathbf{A})^q \mathbf{x} = \sum_{i=1}^d \alpha_i \sigma_i^{2q} \mathbf{v}_i.
\]
Thus, we have
\[
\mathbf{z} = \frac{\sum_{i=1}^d \alpha_i \sigma_i^{2q} \mathbf{v}_i}{\left(\sum_{i=1}^d \alpha_i^2 \sigma_i^{4q}\right)^{1/2}}.
\]

Let \(W\) be the subspace spanned by the singular vectors corresponding to singular values \(\sigma_i \ge (1-\epsilon)\sigma_1\) (say, for \(i=1,\ldots, m\)). Then, the component of \(\mathbf{z}\) perpendicular to \(W\) is given by
\[
\mathbf{z}_\perp = \frac{\sum_{i>m} \alpha_i \sigma_i^{2q} \mathbf{v}_i}{\left(\sum_{i=1}^d \alpha_i^2 \sigma_i^{4q}\right)^{1/2}},
\]
and we have
\[
\|\mathbf{z}_\perp\|^2 = \frac{\sum_{i>m} \alpha_i^2 \sigma_i^{4q}}{\sum_{i=1}^d \alpha_i^2 \sigma_i^{4q}}.
\]

Since for every \(i>m\) we have \(\sigma_i < (1-\epsilon)\sigma_1\), we have
\[
\sum_{i>m} \alpha_i^2 \sigma_i^{4q} \le \sum_{i>m} \alpha_i^2 \left((1-\epsilon)\sigma_1\right)^{4q} \le \left((1-\epsilon)\sigma_1\right)^{4q}.
\]
Since \(|\alpha_1| \ge \eta\), we have
\[
\sum_{i=1}^d \alpha_i^2 \sigma_i^{4q} \ge \alpha_1^2 \sigma_1^{4q} \ge \eta^2 \sigma_1^{4q}.
\]
Thus,
\[
\|\mathbf{z}_\perp\|^2 \le \frac{\left((1-\epsilon)\sigma_1\right)^{4q}}{\eta^2 \sigma_1^{4q}} = \frac{(1-\epsilon)^{4q}}{\eta^2}.
\]

Choose
\[
q = \frac{\log(1/(\epsilon\eta))}{2\epsilon}.
\]
Using the inequality \(1-\epsilon \le e^{-\epsilon}\), we have
\[
(1-\epsilon)^{4q} \le e^{-4q\epsilon} = e^{-4\epsilon\cdot \frac{\log(1/(\epsilon\eta))}{2\epsilon}} = e^{-2\log(1/(\epsilon\eta))} = (\epsilon\eta)^2.
\]
Therefore, we obtain
\[
\|\mathbf{z}_\perp\|^2 \le \frac{(\epsilon\eta)^2}{\eta^2} = \epsilon^2,
\]
and therefore,
\[
\|\mathbf{z}_\perp\| \le \epsilon.
\]

% This shows that the component of \(\mathbf{z}\) perpendicular to \(W\) is at most \(\epsilon\), as required.




    
\end{homeworkProblem}


















\begin{homeworkProblem}
    \solution

%     We wish to show that for a symmetric positive definite matrix 
% \[
% \mathbf{A} \in \mathbb{R}^{d \times d},
% \]
% and random vectors \(x_1,\dots,x_m \in \mathbb{R}^d\) with i.i.d. mean zero sub-Gaussian entries (with fixed sub-Gaussian parameter), the Hutchinson estimator
% \[
% \widetilde{\operatorname{Tr}}_m(\mathbf{A}) = \frac{1}{m} \sum_{l=1}^m x_l^\top \mathbf{A} x_l
% \]
% satisfies
% \[
% \Pr\Bigl[\bigl|\widetilde{\operatorname{Tr}}_m(\mathbf{A}) - \operatorname{Tr}(\mathbf{A})\bigr| \leq \epsilon\, \operatorname{Tr}(\mathbf{A})\Bigr] \geq 1 - \eta,
% \]
% provided that
% \[
% m \geq \frac{c' \log(2/\eta)}{\epsilon^2},
% \]
% for some constant \(c'>0\) (which depends on the sub-Gaussian parameter).

% \subsection*{Step 1. Reformulation via a Block-Diagonal Matrix}

Define the block-diagonal matrix 
\[
\mathbf{B} = \begin{pmatrix}
\mathbf{A} & & \\
& \ddots & \\
& & \mathbf{A}
\end{pmatrix} \in \mathbb{R}^{md \times md},
\]
with \(m\) copies of \(\mathbf{A}\) along the diagonal. Also define the concatenated random vector
\[
z = \begin{pmatrix} x_1 \\ x_2 \\ \vdots \\ x_m \end{pmatrix} \in \mathbb{R}^{md}.
\]
Then we have
\[
z^\top \mathbf{B} z = \sum_{l=1}^m x_l^\top \mathbf{A} x_l,
\]
so that
\[
\widetilde{\operatorname{Tr}}_m(\mathbf{A}) = \frac{1}{m} \, z^\top \mathbf{B} z.
\]
Since each \(x_l\) satisfies
\[
\mathbb{E}\left[x_l^\top \mathbf{A} x_l\right] = \operatorname{Tr}(\mathbf{A}),
\]
we will have
\[
\mathbb{E}\Bigl[z^\top \mathbf{B} z\Bigr] = m\,\operatorname{Tr}(\mathbf{A}).
\]

% \subsection*{Step 2. Applying the Hanson--Wright Inequality}

% The Hanson--Wright inequality (for a symmetric matrix \(\mathbf{B}\) and a random vector \(z\) with i.i.d.\ sub-Gaussian entries) tells us that for any \(t\ge 0\),
% \[
% \Pr\Bigl(\Bigl| z^\top \mathbf{B} z - \mathbb{E}[z^\top \mathbf{B} z] \Bigr| \geq t\Bigr)
% \leq 2 \exp\!\left(-c \min\!\left(\frac{t^2}{\|\mathbf{B}\|_F^2}, \frac{t}{\|\mathbf{B}\|_2}\right)\right),
% \]
% where \(c>0\) is a universal constant (depending on the sub-Gaussian parameter).

Note that due to the block-diagonal structure we have:
\[
\|\mathbf{B}\|_F^2 = m\,\|\mathbf{A}\|_F^2 \quad \text{and} \quad \|\mathbf{B}\|_2 = \|\mathbf{A}\|_2.
\]

We want the estimator to satisfy
\[
\left| \widetilde{\operatorname{Tr}}_m(\mathbf{A}) - \operatorname{Tr}(\mathbf{A}) \right| \leq \epsilon\, \left|\operatorname{Tr}(\mathbf{A})\right|.
\]
Since
\[
\widetilde{\operatorname{Tr}}_m(\mathbf{A}) = \frac{1}{m}\, z^\top \mathbf{B} z,
\]
this is equivalent to
\[
\left| z^\top \mathbf{B} z - m\, \operatorname{Tr}(\mathbf{A}) \right| \leq m\,\epsilon\, |\operatorname{Tr}(\mathbf{A})|.
\]
Thus, setting
\[
t = m\,\epsilon\,|\operatorname{Tr}(\mathbf{A})|,
\]
the Hanson--Wright inequality yields:
\[
\Pr\Bigl(\Bigl| z^\top \mathbf{B} z - m\, \operatorname{Tr}(\mathbf{A}) \Bigr| \geq m\,\epsilon\,|\operatorname{Tr}(\mathbf{A})|\Bigr)
\leq 2 \exp\!\left(- c \min\!\left(\frac{m^2\epsilon^2\,\operatorname{Tr}(\mathbf{A})^2}{m\,\|\mathbf{A}\|_F^2}, \frac{m\,\epsilon\,\operatorname{Tr}(\mathbf{A})}{\|\mathbf{A}\|_2}\right)\right).
\]
equivalently,
\[
\Pr\Bigl(\Bigl| \widetilde{\operatorname{Tr}}_m(\mathbf{A}) - \operatorname{Tr}(\mathbf{A}) \Bigr| \geq \epsilon\, |\operatorname{Tr}(\mathbf{A})|\Bigr)
\leq 2 \exp\!\left(- c \min\!\left(\frac{m\,\epsilon^2\,\operatorname{Tr}(\mathbf{A})^2}{\|\mathbf{A}\|_F^2}, \frac{m\,\epsilon\,\operatorname{Tr}(\mathbf{A})}{\|\mathbf{A}\|_2}\right)\right).
\]

Set
\[
2 \exp\!\left(- c \min\!\left(\frac{m\,\epsilon^2\,\operatorname{Tr}(\mathbf{A})^2}{\|\mathbf{A}\|_F^2}, \frac{m\,\epsilon\,\operatorname{Tr}(\mathbf{A})}{\|\mathbf{A}\|_2}\right)\right)
\leq \eta.
\]
Taking log on both side and we have
\[
c \min\!\left(\frac{m\,\epsilon^2\,\operatorname{Tr}(\mathbf{A})^2}{\|\mathbf{A}\|_F^2}, \frac{m\,\epsilon\,\operatorname{Tr}(\mathbf{A})}{\|\mathbf{A}\|_2}\right)
\geq \log\!\left(\frac{2}{\eta}\right).
\]
meaning
\[
m \geq \frac{1}{c} \max\!\left\{ \frac{\|\mathbf{A}\|_F^2}{\epsilon^2\,\operatorname{Tr}(\mathbf{A})^2}, \frac{\|\mathbf{A}\|_2}{\epsilon\,\operatorname{Tr}(\mathbf{A})} \right\} \log\!\left(\frac{2}{\eta}\right).
\]
Reorganize and absorb the constants into \(c'\) and we have
\[
m \geq \frac{c' \log(2/\eta)}{\epsilon^2},
\]
then
\[
\Pr\Bigl[\bigl|\widetilde{\operatorname{Tr}}_m(\mathbf{A}) - \operatorname{Tr}(\mathbf{A})\bigr| \leq \epsilon\, |\operatorname{Tr}(\mathbf{A})|\Bigr] \geq 1 - \eta.
\]

% \subsection*{Conclusion}

% By considering the block-diagonal matrix and applying the Hanson--Wright inequality with the choice \(t = m\,\epsilon\,\operatorname{Tr}(\mathbf{A})\), we have shown that the Hutchinson estimator is \(\epsilon\)-accurate (relative to \(\operatorname{Tr}(\mathbf{A})\)) with probability at least \(1-\eta\) provided
% \[
% m \geq \frac{c' \log(2/\eta)}{\epsilon^2},
% \]
% which completes the proof.





\end{homeworkProblem}



% We conducted extensive experiments on the MNIST dataset using three different optimization methods: Mixed, Dynamic Barrier, and PCGrad. For each method, we trained a multi-layer perceptron (MLP) with architecture [784-256-128-10] using the Adam optimizer with a learning rate of 3e-4. All experiments were conducted in the full-batch training regime (batch\_size = -1) for 100 epochs. We evaluated four different auxiliary loss functions: L0.5 norm regularization, test loss as auxiliary objective, L2 norm regularization, and Lipschitz constant upper bound. For each configuration, 
% we explored nine different trade-off parameters $\lambda$ logarithmically spaced between $10^{-4}$ and $10^0$, allowing us to thoroughly investigate the Pareto front of the competing objectives. For all methods, we set the exponential moving average (EMA) parameter to 1, and initialized the auxiliary gradient using the "grad" initialization scheme. The experiments were designed to compare the effectiveness of different multi-objective optimization approaches across various regularization schemes. For each configuration, we tracked multiple metrics including training loss, test accuracy, auxiliary loss values, and gradient-related metrics (main gradient norm, projection norm, and orthogonal component norm) to provide comprehensive insights into the optimization dynamics.













\begin{homeworkProblem}
    \solution

    \textbf{Part (a)}
    
    % From the tensor slices:
    \[
    \mathcal{A}_{2,:,:} = 
    \begin{bmatrix}
    \mathcal{A}_{2,:,1} \\
    \mathcal{A}_{2,:,2}
    \end{bmatrix}
    = 
    \begin{bmatrix}
    8 & 2 & 3 \\
    5 & 6 & 4
    \end{bmatrix}
    \quad \text{and} \quad
    \mathcal{A}_{2,3,:} = 
    \begin{bmatrix}
    \mathcal{A}_{2,3,1} \\
    \mathcal{A}_{2,3,2}
    \end{bmatrix}
    =
    \begin{bmatrix}
    3 \\
    4
    \end{bmatrix}
    \]

   
    
    \textbf{Part (b) }

    % We vectorize the tensor by stacking each frontal slice column-wise and then stacking the slices:
    \[
    \mathrm{vec}(\mathcal{A}) =
    \begin{bmatrix}
    3 \\ 8 \\ 4 \\ 9 \\ 2 \\ 3 \\ 1 \\ 3 \\ 9 \\
    6 \\ 5 \\ 1 \\ 9 \\ 6 \\ 4 \\ 5 \\ 4 \\ 7
    \end{bmatrix}
    \]


    
    \textbf{Part (c) }
    The mode-2 unfolding $\mathbf{A}_{(2)}$ treats rows as mode-2 fibers:
    \[
    \mathbf{A}_{(2)} = 
    \begin{bmatrix}
    3 & 8 & 4 & 6 & 5 & 1 \\
    9 & 2 & 3 & 9 & 6 & 4 \\
    1 & 3 & 9 & 5 & 4 & 7
    \end{bmatrix}
    \]

    The mode-3 unfolding $\mathbf{A}_{(3)}$ stacks each $i,j$ location across slices:
    \[
    \mathbf{A}_{(3)} = 
    \begin{bmatrix}
    3 & 8 & 4 & 9 & 2 & 3 & 1 & 3 & 9 \\
    6 & 5 & 1 & 9 & 6 & 4 & 5 & 4 & 7
    \end{bmatrix}
    \]



    \textbf{Part (d) }
    
    \[
    \|\mathcal{A}\|_F^2 = 
    3^2 + 9^2 + 1^2 + 8^2 + 2^2 + 3^2 + 4^2 + 3^2 + 9^2 + 
    6^2 + 9^2 + 5^2 + 5^2 + 6^2 + 4^2 + 1^2 + 4^2 + 7^2
    \]

    \[
    = 559
    \]

\end{homeworkProblem}

















\begin{homeworkProblem}
    \solution

    \textbf{Part (a) }

    Let
\[
A = [a_1, a_2, \dots, a_n] \quad \text{and} \quad B = [b_1, b_2, \dots, b_n],
\]
where \(a_i\) and \(b_i\) denote the \(i\)-th columns of \(A\) and \(B\). 
% The Khatri-Rao product is defined as the columnwise Kronecker product:
% \[
% B \odot A = \begin{bmatrix} b_1 \otimes a_1 & b_2 \otimes a_2 & \cdots & b_n \otimes a_n \end{bmatrix}.
% \]

% \subsection*{1. Proof that \((B \odot A)^\top (B \odot A) = B^\top B \ast A^\top A\)}

The \((i,j)\)-th entry of \((B \odot A)^\top (B \odot A)\) is
\[
\left[(B \odot A)^\top (B \odot A)\right]_{ij} = (b_i \otimes a_i)^\top (b_j \otimes a_j).
\]
Using the property of the Kronecker product,
% \[
% (x \otimes y)^\top = x^\top \otimes y^\top,
% \]
we have
\[
(b_i \otimes a_i)^\top (b_j \otimes a_j) = \left(b_i^\top \otimes a_i^\top\right)(b_j \otimes a_j).
\]
Then using the other property of the Kronecker product,
\[
(b_i^\top \otimes a_i^\top)(b_j \otimes a_j) = (b_i^\top b_j)(a_i^\top a_j).
\]
Thus, the \((i,j)\)-th entry is
\[
\left[(B \odot A)^\top (B \odot A)\right]_{ij} = (b_i^\top b_j)(a_i^\top a_j).
\]
This is exactly the \((i,j)\)-th entry of the Hadamard (elementwise) product \(B^\top B \ast A^\top A\). Hence,
\[
(B \odot A)^\top (B \odot A) = B^\top B \ast A^\top A.
\]
% \qed
% \subsection*{2. Proof that \((B \otimes A)(D \odot C) = (BD) \odot (AC)\)}

\textbf{Part (b) }

Let
\[
C = [c_1, c_2, \dots, c_n] \quad \text{and} \quad D = [d_1, d_2, \dots, d_n],
\]
so that the Khatri-Rao product is given by
\[
D \odot C = \begin{bmatrix} d_1 \otimes c_1 & d_2 \otimes c_2 & \cdots & d_n \otimes c_n \end{bmatrix}.
\]
Consider the product of \((B \otimes A)\) with the \(k\)-th column of \((D \odot C)\):
\[
(B \otimes A)(d_k \otimes c_k).
\]
By the properties of the Kronecker product, we have
\[
(B \otimes A)(d_k \otimes c_k) = (B d_k) \otimes (A c_k).
\]
Since this holds for each column \(k = 1, 2, \dots, n\), it follows that
\[
(B \otimes A)(D \odot C) = \begin{bmatrix} (B d_1) \otimes (A c_1) & (B d_2) \otimes (A c_2) & \cdots & (B d_n) \otimes (A c_n) \end{bmatrix}.
\]
By the definition of the Khatri-Rao product, the right-hand side is exactly
\[
(BD) \odot (AC),
\]
since the \(k\)-th column of \((BD) \odot (AC)\) is \((B d_k) \otimes (A c_k)\). Therefore,
\[
(B \otimes A)(D \odot C) = (BD) \odot (AC).
\]


    
\end{homeworkProblem}



\begin{homeworkProblem}
    \solution

    The work is in the Colab notebook.
    

\end{homeworkProblem}






























\end{document}
